%% ------------------------------------------------------------------------- %%
\chapter{Conclusão}
\label{cap:conclusoes}

	O aumento da qualidade de imagens proveniente da melhoria da tecnologia associada com cameras e afins, juntamente com
a crescente facilidade de manter e utilizar grande quantidade de dados melhorou a condição para a realização de estudos
envolvendo uma quantidade gigantesca de dados proveniente de imagens. Para acelerar o processamento dessa
grande quantidade de dados estudamos algoritmos de registro que fossem facilmente paralelizaveis.

	A escolha da GPU como plataforma utilizada para acelerar os algoritmos veem da velocidade com a qual a sua capacidade
de processamento está aumentando. Se comparadas com placas de três anos atrás, GPUs de hoje em dia apresentam uma melhoria
de, em torno, três a cinco vezes mais poder de processamento. Com esses requisitos em mente estudamos algoritmos de registro
que se adaptam à arquitetura SIMD sem precisar de muitas modificações. Os dois escolhidos para passarem em testes de
eficácia foram o \textit{Thin Plate Splines} e o \textit{Demons}.

	Nossos testes apontam para a escolha do TPS como o algoritmo que será portado para a plataforma GPGPU. Seu algoritmo
pode ser escrito fácilmente seguinto o SIMD e seus resultados para deformações mais gerais foram superiores ao 
\textit{Demons}.

%------------------------------------------------------
\section{Próximos Passos} 

	Os próximos passos para completar o estudo são, em prioridade:
\begin{itemize}
	\item[\textbf{A}] Escolher e utilizar algum algoritmo para encontrar e corresponder as características nas imagens;
	\item[\textbf{B}] Criar um arcabouço em torno do TPS para facilitar a sua execução com multiplas entradas;
	\item[\textbf{C}] Criar uma versão GPGPU do TPS utilizando o CUDA;
	\item[\textbf{D}] Estudar e implementar um método para acelerar para o algoritmo escolhido no passo 1;
	\item[\textbf{E}] Escrever a dissertação.
\end{itemize}

\section{Cronograma}

\begin{table}
\begin{center}
\begin{small}
\begin{tabular}{|c|c|c|c|c|c|c|c|} 
\hline
\emph{Tarefa} &
Março & 
Abril & 
Maio &  
Junho & 
Julho &
Agosto & 
Setembro \\ \hline
\textbf{A} & X & X & X & X & X & X & X \\ \hline 
\textbf{B} & X & X & X & X & X & X & X \\ \hline 
\textbf{C} & X & X & X & X & X & X & X \\ \hline 
\textbf{D} & X & X & X & X & X & X & X \\ \hline 
\textbf{E} & X & X & X & X & X & X & X \\ \hline 
\end{tabular}
\caption{Cronograma.}
\label{tab:tab:F5}
\end{small}
\end{center}
\end{table}