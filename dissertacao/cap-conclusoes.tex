%% ------------------------------------------------------------------------- %%
\chapter{Trabalhos futuros}
\label{cap:conclusoes}

	O aumento da qualidade de imagens proveniente da melhoria da tecnologia associada com câmeras e afins, juntamente com
a crescente facilidade de manter e utilizar grande quantidade de dados melhorou a condição para a realização de estudos
envolvendo uma quantidade gigantesca de dados proveniente de imagens. Para acelerar o processamento dessa
grande quantidade de dados estudamos algoritmos de registro que fossem facilmente paralelizáveis.

	A escolha da GPU como plataforma utilizada para acelerar os algoritmos veem da velocidade com a qual a sua capacidade
de processamento está aumentando. Se comparadas com placas de três anos atrás, GPUs de hoje em dia apresentam uma melhoria
de, em torno, três a cinco vezes mais poder de processamento. Com esses requisitos em mente estudamos algoritmos de registro
que se adaptam à arquitetura SIMD sem precisar de muitas modificações. Os dois escolhidos para passarem em testes de
eficácia foram o \textit{Thin Plate Splines} e o \textit{Demons}.
%------------------------------------------------------
\section{Próximos Passos} 

	Os próximos passos para completar o estudo são, em prioridade:
\begin{enumerate}
	\item Avaliar algoritmos utilizados na etapa 1 e 2 do registro para utilizá-los;
	\item Criar um arcabouço em torno do TPS para facilitar a sua execução com múltiplas entradas;
	\item Criar uma versão GPGPU do TPS utilizando o CUDA;
	\item Estudar e implementar um método para acelerar o algoritmo escolhido no passo 1;
	\item Publicar um artigo;
	\item Escrever a dissertação.
\end{enumerate}

\section{Cronograma}

\begin{table}
\begin{center}
\begin{small}
\begin{tabular}{|c|c|c|c|c|c|c|c|} 
\hline
\emph{Tarefa} &
Março & 
Abril & 
Maio &  
Junho & 
Julho &
Agosto & 
Setembro \\ \hline
\textbf{1} & X &  &  &  &  &  &  \\ \hline 
\textbf{2} & X & X &  &  &  &  &  \\ \hline 
\textbf{3} &  &  & X & X &  &  &  \\ \hline 
\textbf{4} &  &  &  & X & X &  &  \\ \hline 
\textbf{5} & X & X & X &  &  &  &  \\ \hline 
\textbf{6} &  &  & X & X & X & X & X \\ \hline 
\end{tabular}
\caption{Cronograma.}
\label{tab:tab:F5}
\end{small}
\end{center}
\end{table}