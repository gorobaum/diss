%% ------------------------------------------------------------------------- %%
\chapter{Introdução}
\label{cap:introducao}

	Imagens de altissima resolução, chamadas de \textit{Gigapixel}, são imagens construidas utilizando milhares de 
menores imagens. 

	O registro de imagens é uma área bem firmada e amplamente utilizada para alinhar duas ou mais imagens. Os algoritmo 
de registro são amplamente aplicados em várias áreas de pesquisa em visão computacional, como em imagens médicas, com o 
objetivo de reverter as deformações naturais dos técidos moles entre tomadas de imagens de um paciente, ou em 
reconhecimento de padrões, onde o registro é aplicado para construir um mapa a partir de várias imagens 
obtidas, por exemplo, de um satélite.

	O registro é

	Com o crescimento da qualidade das imagens obtidas e a criação de técnicas mais sofisticadas de registro,
o tempo de execução 



Eu tenho que explicar que o registro é utilizado em várias áreas, e que ele leva uma quantidade de tempo
rasoavel para ser executado, dado o tamanho das imagens e a complexidade intrinsica do processo.


Falar sobre
gigapixel - o que é em linhas gerais, onde começou, onde é usado
registro - áreas que usam registro, como ele é usado
registtro + gigapixel - o problema de desempenho, como resolver.

% \emph{Thesis are random access. Do NOT feel obliged to read a thesis from beginning to end.}

%% ------------------------------------------------------------------------- %%
\section{Objetivos}
\label{sec:objetivo}

%% ------------------------------------------------------------------------- %%
\section{Organização do Trabalho}
\label{sec:organizacao_trabalho}

No Capítulo~\ref{cap:conceitos}, apresentamos os conceitos ... Finalmente, no
Capítulo~\ref{cap:conclusoes} discutimos algumas conclusões obtidas neste
trabalho. Analisamos as vantagens e desvantagens do método proposto ... 

As sequências testadas no trabalho estão disponíveis no Apêndice \ref{ape:sequencias}.
