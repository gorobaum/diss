%% ------------------------------------------------------------------------- %%
\chapter{Introdução}
\label{cap:introducao}

	Imagens de altissima resolução, chamadas de \textit{Gigapixel}, são imagens construidas utilizando milhares de 
menores imagens. 

	O registro de imagens é uma área bem firmada e amplamente utilizada para alinhar duas ou mais imagens. Os algoritmo 
de registro são amplamente aplicados em várias áreas de pesquisa em visão computacional, como em imagens médicas, com o 
objetivo de reverter as deformações naturais dos técidos moles entre tomadas de imagens de um paciente, ou em 
reconhecimento de padrões, onde o registro é aplicado para construir um mapa a partir de várias imagens 
obtidas, por exemplo, de um satélite.

	O registro é

	Com o crescimento da qualidade das imagens obtidas e a criação de técnicas mais sofisticadas de registro,
o tempo de execução 



Eu tenho que explicar que o registro é utilizado em várias áreas, e que ele leva uma quantidade de tempo
rasoavel para ser executado, dado o tamanho das imagens e a complexidade intrinsica do processo.


    Para criar a imagem \textit{Gigapixel} um algoritmo de registro é executado $(n_c-1)*(n_l-1)$ vezes no minimo, 
onde $n_c$ e $n_l$ são o número de colunas e linhas de fotografias que compõem a imagem \textit{Gigapixel}, 
respectivamente. Esse número é uma estimativa otimista, já que mais processos de registro são potencialmente executados
entre fragmentos para melhorar a qualidade da imagem final. Uma das maiores imagens \textit{Gigapixel} até hoje, 
o projeto \textit{Servilla 111 Gigapixels}, coordenado por \cite{sevilla111},  conta com 9,750 fotografias, dispostas
em 65 linhas. Nesse projeto foram executados, no melhor dos casos, 9536 registros entre fotos de 22 \textit{Megapixels}
, algo claramente custoso.

    Podemos acelerar o processo de custura das imagens usando várias abordagens diferentes. A mais clássica seria
paralelizar o algoritmo inteiro, utilizando a grande capacidade das \textit{CPUs} atuais em executar processos paralelos.
Outra abordagem é atacar diretamente o passo do registro, construindo algoritmos menos custosos ou melhorar algoritmos
existentes. A abordagem escolhida foi modificar o TPS, criando uma versão que segue a arquitetura 
\textit{Single Instruction Multiple Data} (SIMD). A escolha foi feita pensando em uma implementação para 
Unidades de Processamento Gráfico (\textit{Graphic Processing Units} (GPUs)).



Falar sobre
gigapixel - o que é em linhas gerais, onde começou, onde é usado
registro - áreas que usam registro, como ele é usado
registtro + gigapixel - o problema de desempenho, como resolver.

% \emph{Thesis are random access. Do NOT feel obliged to read a thesis from beginning to end.}

%% ------------------------------------------------------------------------- %%
\section{Objetivos}
\label{sec:objetivo}

%% ------------------------------------------------------------------------- %%
\section{Trabalhos Relacionados}
\label{sec:objetivo}

%% ------------------------------------------------------------------------- %%
\section{Organização do Trabalho}
\label{sec:organizacao_trabalho}

No Capítulo~\ref{cap:conceitos}, apresentamos os conceitos ... Finalmente, no
Capítulo~\ref{cap:conclusoes} discutimos algumas conclusões obtidas neste
trabalho. Analisamos as vantagens e desvantagens do método proposto ... 

As sequências testadas no trabalho estão disponíveis no Apêndice \ref{ape:sequencias}.
