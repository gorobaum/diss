%% ------------------------------------------------------------------------- %%
\chapter{Introdução}
\label{cap:introducao}

  O alinhamento espacial ou geométrico entre duas imagens é um problema clássico
de visão computacional, chamado de registro de imagens. O registro é aplicado
para alinhar imagens da mesma cena oriundas de diferentes sensores, adquiridas
em diferentes intervalos ou capturadas de diferentes pontos de vista. O resultado
do registro é uma imagem que contém o alinhamento de duas imagens diferentes, somando
as informações contidas em cada uma das imagens em um mesmo espaço. O registro é
usado para identificar modificações em cenas, alinhar imagens ou encontrar objetos.

  No estudo e processamento de imagens tomográficas, o registro é aplicados
com as seguintes finalidades: (i) na junção de informações de imagens de 
diferentes modalidades, como a sobreposição de imagens de ressonância magnética 
com imagens de raio-x; (ii) estudos envolvendo atlas anatômicos, onde cada 
paciente do estudo tem seus dados registrados para um modelo de estudo; 
(iii) corrigir diferenças de imagens de pré- e pós-operatório, causadas 
por movimentações do paciente e por movimentação de tecido mole, como a 
movimentação do pulmão na respiração.

  O registro deve ser capaz de modelar fielmente as deformações sofridas
pelos tecidos do corpo humano, sejam elas rígidas, como a movimentação
de um osso ou membro, ou não rígidas, como a movimentação de tecidos mole.
O estudo, então, se restringe á algoritmos de registro que utilizem funções
de alinhamento com componentes rígidas e não rígidas.Um dos algoritmos mais utilizados 
pela comunidade cientifica para resolver casos de registro não-rígidos é 
o \textbf{Thin Plate Splies} (TPS) \cite{goshtasby1988registration} e 
\cite{bookstein1989principal}. O TPS é facilmente aplicavel a uma gama
de problemas, já que ele só necessita de uma relação entre conjuntos de pontos
das duas imagens, e dado que sua função de alinhamento é composta de 
uma componente rígida e outra não rígida. Mas essa maleabilidade vem com
um custo, o alto tempo de execução, que é fortemente influenciado pelo
tamanho do conjunto de pontos usado como entrada. Esse dilema cria uma
busca por um equilíbrio no tamanho desse conjunto, já que um número pequeno
de correspondências resulta em uma imagem mal alinhada, e um número muito
grande eleva muito o tempo de execução.

  A fim de apresentar um resultado de qualidade dentro de um espaço de tempo
razoável, o trabalho apresenta uma implementação do TPS no paradigma Programação 
Genérica em Unidades de  Processamento Gráfico (GPGPU). Para atingir um tempo
de execução realista, artifícios ligados a arquitetura das GPUs foram utilizados,
como a otimização de utilização de recursos da GPU, mapeamento de imagens para
texturas e execução concorrente de instâncias diferentes de registro. Para a
implementação do código, a linguagem CUDA foi escolhida, a partir dos resultados
do estudo de \cite{membarth2011frameworks}.
%% ------------------------------------------------------------------------- %%
\section{Trabalhos Relacionados}
\label{sec:trabalhosRelacionados}

  O TPS foi introduzido por \cite{bookstein1989principal} e \cite{goshtasby1988registration}.
O trabalho de \citeauthor{goshtasby1988registration} introduz o TPS como um método
de registro não rígido. \citeauthor{bookstein1989principal} argumenta que as splines 
utilizadas pela componente não rígida do TPS compõem um bom modelo para representar
deformações em tecidos vivos. Ele ainda realiza um extenso estudo sobre a
construção do conjunto de relações entre os pontos de controle em cada uma das imagens
no resultado do registro. Ele demonstra que, principalmente, a localidade e o 
espaçamento entre os pontos escolhidos tem grande influência na qualidade do resultado.

  \cite{rohr1996nonrigid} introduz um fator de suavização à função de transformação
do TPS. Esse fator melhora o resultado quando a correspondência entre os pontos de controle
não é exata, seja por presença de ruido ou por má marcação dos pontos. Em uma continuação 
do seu trabalho, \cite{rohr2001landmark} acrescenta uma modelagem para pontos de controles
mal definidos, adicionando uma aproximação quadrática quando a posição
de algum ponto de controle é usado. Para melhorar o tempo de execução do TPS,
\cite{flusser1992adaptive} propõe a aplicação do TPS em sub regiões quadradas ou
triangulares das imagens, diminuindo o número de pontos de controle e cada uma
das instâncias do TPS. \cite{beatson1992fast} aplica expansões hierárquicas na
função de transformação do TPS para reduzir o tempo de execução final. Esse
método calcula explicitamente a influência de pontos de controles próximos,
enquanto simplifica a interação de pontos de controles distantes. Com isso,
o tempo de execução da influência de um ponto de controle sobre os outros cai de
$\mathcal{O}(n_{cp}^2)$ para $\mathcal{O}(n_{cp}\log{n_{cp}})$. Ainda sim, a
eficiência do TPS é limitada pelo número de pontos de controle escolhidos.

\subsection{Aceleração de registro por GPGPU}

  A utilização de GPUs para agilizar o processo de registro é algo comum
atualmente, principalmente pela facilidade e mapear problemas de visão
computacional para GPUs. \cite{strzodka2004image} foi um dos primeiros artigos
a portar um algoritmo de registro não rígido para GPUs. Nele, um método discreto
de minimização de fluxo de gradiente foi implementado em GPGPU, utilizado o
\textbf{DirectX 9} para a implementação. O método foi aplicado no registro
de imagens cerebrais. \cite{kohn2006gpu} expandiu o trabalho
anterior, criando uma versão do algoritmo para imagens em três dimensões, 
especialmente imagens tomográficas, realizando uma etapa de registro rígido antes 
do não rígido. \cite{vetter2007non} realiza o mapeamento de um algoritmo de 
registro multimodal para imagens médicas. Esse trabalho é um dos primeiros a 
explicar como os recursos da GPU foram utilizados no mapeamento, como por 
exemplo a organização dos dados na memória. O trabalho de \cite{grossauer2008gpu} 
relata a criação de um algoritmo para registro que utiliza fluxo óptico em conjunto 
com métodos de \textit{Multi Grid}, usados para resolver equações diferenciais 
parciais. Ele descreve a aceleração do \textit{Multi Grid}, dada a sua 
afinidade com o \textit{Pipeline} gráfico.

  Em \cite{bui2009performance} o registro rígido de imagens multimodais médicas
utilizando fluxo óptico. A implementação foi realizada em CUDA
\cite{nvidia2007compute}. \cite{han2009gpu} utiliza Informação Mutua para
realizar um registro não rígido de imagens cerebrais de ressonância magnética
e 17 atlas médicos, com a finalidade de segmentar regiões de interesse nos
pacientes.Em 2010, o artigo de \cite{modat2010fast} define um algoritmo de
registro não rígido chamado de \textit{Free-Form Deformation}, desenvolvido
para registrar imagens de ressonância magnética T1. Diferente de outros
algoritmos, ele foi desenvolvido para executar em GPUs diretamente. Para tal,
os autores utilizam uma interpolação B-Spline cúbica e uma modificação da
informação mutua como métrica para o registro. Ele separa cara passo em vários
\textit{kernels} diferentes, fazendo com que cada \textit{kernel} utilize o
máximo de memória possível e então espalha o processamento para um maior número
de processadores na GPU, melhorando o desempenho do algoritmo. Para comparar o
resultado, uma implementação em C++ foi feita. O estudo de
\cite{membarth2011frameworks} realiza uma comparação entre vários arcabouços de
programação para GPU, entre eles o CUDA e o OpenCL utilizando como caso de
comparação o registro de imagens 2D/3D, ou seja, o registro de imagens em três
dimensões de tomografia computadorizada para imagens de raio-X. O artigo então
segue primeiro com uma comparação conceitual entre os vários arcabouços e
depois com uma comparação de desempenho. Foram feitas comparações com a
execução sequencial, em paralelo e em GPU do algoritmo.

\begin{table}[]
    \begin{tabular}{| l | l | l | p{6em} |}
      \hline
      Artigo & Hardware (Implementação) & Tamanho da Imagem & Tempo de Execução (Aceleração) \\ \hline
      \cite{strzodka2004image}\ding{108} & GeForceFX 5800 (DX9) & $513 \times 769$ & 8.3s (4x) \\ \hline
      \cite{kohn2006gpu}\ding{108} & GeForce 6800 (DX9) & $256 \times 256 \times 128$ & 5s (12x) \\ \hline
      \cite{vetter2007non}\ding{108} & GeForce 7800 GTX (DX10) & $512 \times 512$ & 3.9s (6x) \\ \hline
      \cite{bui2009performance}\ding{109} & Tesla C870 (CUDA) & $512 \times 512$ & 0.69s (7.48x) \\ \hline
      \cite{han2009gpu}\ding{108} & GeForce GTX 280 (CUDA) & $256 \times 256 \times 128$ & 19s (25x) \\ \hline
      \cite{modat2010fast}\ding{108} & GeForce 8800 GTX (CUDA) & $181 \times 217 \times 181$ & 42s (9.89x) \\ \hline
      \cite{luo2014gpu}\ding{108} & Tesla C2050 (CUDA) & $256 \times 256 \times 94$ & 0.252s (69x) \\ \hline
    \end{tabular}
    \caption{Tabela contendo o hardware, tamanho da imagem e aceleração dos artigos citados.
             \ding{108} registro não rígido. \ding{109} registro rígido.}
\end{table}
  

  Pouco avanço foi feito na melhoria do desempenho do TPS utilizando GPU. A
grande parte dos trabalhos emprega um mapeamento trivial, sem entrar em detalhes
sobre a implementação nem evidenciar a utilização de recursos da GPU para tentar
melhorar a implementação. Tanto \cite{richa2011towards} quanto
\cite{schoob2013stereoscopic} apresentam algoritmos usados por câmeras no
auxilio em operações cardiovasculares. Para operar em tempo real, requisito
minimo para esse tipo de sistema, os dois trabalhos utilizam implementações
em GPUs NVIDIA do TPS, mas sem entrar em detalhes sobre o mapeamento feito.
\cite{luo2014gpu} apresenta um mapeamento completo do TPS utilizando CUDA,
explicando como cada etapa do algoritmo é implementado.

%% ------------------------------------------------------------------------- %%
\section{Objetivos}

  Os trabalhos apresentados acima não desfrutam de nenhum método para melhorar
o desempenho de suas aplicações em CUDA. Este trabalho tem como objetivo aplicar
e confirmar o impacto de métodos para melhorar o desempenho do TPS em GPGPU.
Esses métodos serão aplicados na implementação do TPS voltada para solução de 
registros para atlas médicos, onde várias imagens diferentes são registradas 
para a mesma imagem referência. Os objetivos específicos do trabalho são:

\begin{enumerate}
  \item Desenvolver uma implementação para GPGPU que contemple os seguintes métodos
        de melhora de desempenho: (i) execução concorrente de kernels; 
        (ii) utilização de textura tridimensional;
        (iii) cálculo de ocupação de recursos da GPU

  \item Avaliar o desempenho de cada uma das melhorias aplicadas
  
  \item Avaliar o desempenho da aplicação resultante no registro de imagens médicas tomográficas
\end{enumerate}

%% ------------------------------------------------------------------------- %%
\section{Organização do Trabalho}
\label{sec:organizacao_trabalho}

No Capítulo~\ref{cap:conceitos}, os conceitos utilizados pelo trabalho
são apresentados, como registro e GPGPU. O capítulo \ref{cap:metodologia}
mostra como os conceitos são combinados para a criação da implementação que
satisfaz nossos objetivos. No Capítulo \ref{cap:resultados} os experimentos são
descritos e os seus resultados avaliados. Finalmente, no Capítulo
~\ref{cap:conclusao} analisamos os resultados obtidos.
