%% ------------------------------------------------------------------------- %%
\chapter{Introdu��o}
\label{cap:introducao}

	Imagens de altissima resolu��o, chamadas de \textit{Gigapixel}, s�o imagens construidas utilizando milhares de 
menores imagens. 

	O registro de imagens � uma �rea bem firmada e amplamente utilizada para alinhar duas ou mais imagens. Os algoritmo 
de registro s�o amplamente aplicados em v�rias �reas de pesquisa em vis�o computacional, como em imagens m�dicas, com o 
objetivo de reverter as deforma��es naturais dos t�cidos moles entre tomadas de imagens de um paciente, ou em 
reconhecimento de padr�es, onde o registro � aplicado para construir um mapa a partir de v�rias imagens 
obtidas, por exemplo, de um sat�lite.

	O registro �


	Com o crescimento da qualidade das imagens obtidas e a cria��o de t�cnicas mais sofisticadas de registro,
o tempo de execu��o 



Eu tenho que explicar que o registro � utilizado em v�rias �reas, e que ele leva uma quantidade de tempo
rasoavel para ser executado, dado o tamanho das imagens e a complexidade intrinsica do processo.


Falar sobre
gigapixel - o que � em linhas gerais, onde come�ou, onde � usado
registro - �reas que usam registro, como ele � usado
registtro + gigapixel - o problema de desempenho, como resolver.

% \emph{Thesis are random access. Do NOT feel obliged to read a thesis from beginning to end.}

%% ------------------------------------------------------------------------- %%
\section{Objetivos}
\label{sec:objetivo}

ME QUALIFICAR

%% ------------------------------------------------------------------------- %%
\section{Organiza��o do Trabalho}
\label{sec:organizacao_trabalho}

No Cap�tulo~\ref{cap:conceitos}, apresentamos os conceitos ... Finalmente, no
Cap�tulo~\ref{cap:conclusoes} discutimos algumas conclus�es obtidas neste
trabalho. Analisamos as vantagens e desvantagens do m�todo proposto ... 

As sequ�ncias testadas no trabalho est�o dispon�veis no Ap�ndice \ref{ape:sequencias}.
