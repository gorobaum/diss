%% ------------------------------------------------------------------------- %%
\chapter{Introdução}
\label{cap:introducao}

% \emph{Thesis are random access. Do NOT feel obliged to read a thesis from beginning to end.}

%% ------------------------------------------------------------------------- %%
\section{Trabalhos Relacionados}
\label{sec:objetivo}
	Desde que as Unidades de Processamento Gráfico começaram a ser usadas em meados dos anos 2002 para execução de 
código genérico, ou seja, sem a finalidade de renderizar uma cena, vários trabalhos traduziram algoritmos para a GPU.
Algoritmos esses que são aplicados em várias áreas, como cálculo númerico, simulações cientificas e processamento de 
imagens.

	Uma das primeira aplicações envolvendo o uso de GPUs no processamento de imagens foi feito por 
\cite{fung2005openvidia}, que descreve o arcabouço de programação \textit{OpenVIDIA}, criado a partir do 
\textit{OpenGL}, desenvolvido por \cite{opengl}. Antes do \textit{OpenVIDIA} o processamento de imagens em GPUs era
feito utilizando o \textit{OpenGL}, uma linguagem para processamento gráfico, voltada para a renderização de cenas. Os
algoritmos para processamento de imagem eram reescritos para utilizarem primitivas do \textit{OpenGL}. O 
\textit{OpenVIDIA} criou um encapsulamento de chamadas do \textit{OpenGL} voltado para imagens, retirando a necessidade
de traduzir os alçgoritmos e facilitando a utilização dos \textit{shaders} programáveis da GPU.

	Com a introdução de linguagens próprias para programação em GPU, vários algoritmos de processamento de imagens foram
traduzidos para a execução em GPUs. Na área de registro um dos primeiros artigos a tratar do assunto foi 
\cite{strzodka2004image}, que apresenta uma implementação de um algoritmo de registro não rigido que utiliza uma técnica
de redução de energia da diferença das imagens. \cite{kohn2006gpu} expandiu o trabalho anterior, criando uma versão do 
algoritmo para imagens em três dimensões e realizando um registro rigido antes do não-rigido. Ainda programando 
diretamente em \textit{OpenGL}, \cite{vetter2007non} propõem um algoritmo de registro multimodal para imagens médicas 
executando em uma GPU 7800 da \textit{NVIDIA}. O artigo leva em conta a organização da imagem na memória da GPU e o 
tempo de transmissão dos dados da CPU para a GPU.

	O trabalho de \cite{grossauer2008gpu} relata a criação de um algoritmo para registro que utiliza fluxo optico em 
conjunto com métodos de \textit{Multi Grid}, usados para resolver equações diferenciais parciais. Ele descreve
a aceleração do \textit{Multi Grid}, dada a sua afinidade com o \textit{Pipeline} gráfico. Em \cite{bui2009performance}
o passo de interpolação de um algoritmo de registro multi modal entre imagens médicas é traduzido utilizando uma 
linguagem para programação genérica em GPUs, o CUDA \cite{nvidia2007compute}. O estudo realiza uma comparação de 
eficiência entre o código em CPU e GPU de uma interpolação bilinear, apresentando como resultado uma aceleração de 60 
vezes da GPU em relação a CPU.
	
%% ------------------------------------------------------------------------- %%
\section{Motivações}

%% ------------------------------------------------------------------------- %%
\section{Objetivos}

%% ------------------------------------------------------------------------- %%
\section{Organização do Trabalho}
\label{sec:organizacao_trabalho}

No Capítulo~\ref{cap:conceitos}, apresentamos o que é Registro, Unidade de Processamento Gráfico, GPGPU e os algoritmos 
de registro escolhidos. No Capítulo \ref{cap:resultados} os testes utilizados e os resultados dos dois algoritmos 
escolhidos são apresentados. Finalmente, no Capítulo~\ref{cap:conclusoes} analisamos os resultados obtidos 