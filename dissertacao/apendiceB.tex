\chapter{Kernels da implementação do TPS}\index{kernelCode}\label{kernelCode}

\begin{lstlisting}[label={code:tpsBasic}, caption={Kernel sem nenhum tipo de melhoria}]
  __global__ void tpsCuda(short* cudaImage, short* cudaRegImage,
                          float* solutionX, float* solutionY,
                          float* solutionZ, int width, int height,
                          int slices, float* keyX, float* keyY,
                          float* keyZ, int numOfKeys) {
    int x = blockDim.x*blockIdx.x + threadIdx.x;
    int y = blockDim.y*blockIdx.y + threadIdx.y;
    int z = blockDim.z*blockIdx.z + threadIdx.z;

    float newX = solutionX[0]+x*solutionX[1]+y*solutionX[2]+z*solutionX[3];
    float newY = solutionY[0]+x*solutionY[1]+y*solutionY[2]+z*solutionY[3];
    float newZ = solutionZ[0]+x*solutionZ[1]+y*solutionZ[2]+z*solutionZ[3];

    for (int i = 0; i < numOfKeys; i++) {
      float r = (x-keyX[i])*(x-keyX[i])+  (y-keyY[i])*(y-keyY[i])+(z-keyZ[i])*(z-keyZ[i]);
      if (r != 0.0) {
        newX += r*log(r)*solutionX[i+4];
        newY += r*log(r)*solutionY[i+4];
        newZ += r*log(r)*solutionZ[i+4];
      }
    }

    if (x <= width-1 && x >= 0)
      if (y <= height-1 && y >= 0)
        if (z <= slices-1 && z >= 0)
          cudaRegImage[z*height*width+y*width+x] = cudaTrilinearInterpolation(newX, newY, newZ, cudaImage, width, height, slices);
  }
\end{lstlisting}

\begin{lstlisting}[label={code:tpsTexture}, caption={Kernel utilizando mapeamento da imagem para textura}]
__global__ void tpsCudaWithText(cudaTextureObject_t textObj, short* cudaRegImage, float* solutionX, float* solutionY,
                        float* solutionZ, int width, int height, int slices, float* keyX, float* keyY,
                        float* keyZ, int numOfKeys) {
  int x = blockDim.x*blockIdx.x + threadIdx.x;
  int y = blockDim.y*blockIdx.y + threadIdx.y;
  int z = blockDim.z*blockIdx.z + threadIdx.z;

  float newX = solutionX[0] + x*solutionX[1] + y*solutionX[2] + z*solutionX[3];
  float newY = solutionY[0] + x*solutionY[1] + y*solutionY[2] + z*solutionY[3];
  float newZ = solutionZ[0] + x*solutionZ[1] + y*solutionZ[2] + z*solutionZ[3];

  for (int i = 0; i < numOfKeys; i++) {
    float r = (x-keyX[i])*(x-keyX[i]) + (y-keyY[i])*(y-keyY[i]) + (z-keyZ[i])*(z-keyZ[i]);
    if (r != 0.0) {
      newX += r*log(r) * solutionX[i+4];
      newY += r*log(r) * solutionY[i+4];
      newZ += r*log(r) * solutionZ[i+4];
    }
  }

  if (x <= width-1 && x >= 0)
    if (y <= height-1 && y >= 0)
      if (z <= slices-1 && z >= 0)
        cudaRegImage[z*width*height+y*width+x] = (short)tex3D<float>(textObj, newX, newY, newZ);
}
\end{lstlisting}

\begin{lstlisting}[label={code:tpsNoInter}, caption={Kernel sem a interpolação trilinear}]
__global__ void tpsCudaWithoutInterpolation(float* cudaPointsX, float* cudaPointsY, float* cudaPointsZ, float* solutionX, float* solutionY,
                        float* solutionZ, int width, int height, int slices, float* keyX, float* keyY,
                        float* keyZ, int numOfKeys) {
  int x = blockDim.x*blockIdx.x + threadIdx.x;
  int y = blockDim.y*blockIdx.y + threadIdx.y;
  int z = blockDim.z*blockIdx.z + threadIdx.z;

  float newX = solutionX[0] + x*solutionX[1] + y*solutionX[2] + z*solutionX[3];
  float newY = solutionY[0] + x*solutionY[1] + y*solutionY[2] + z*solutionY[3];
  float newZ = solutionZ[0] + x*solutionZ[1] + y*solutionZ[2] + z*solutionZ[3];

  for (int i = 0; i < numOfKeys; i++) {
    float r = (x-keyX[i])*(x-keyX[i]) + (y-keyY[i])*(y-keyY[i]) + (z-keyZ[i])*(z-keyZ[i]);
    if (r != 0.0) {
      newX += r*log(r) * solutionX[i+4];
      newY += r*log(r) * solutionY[i+4];
      newZ += r*log(r) * solutionZ[i+4];
    }
  }

  if (x <= width-1 && x >= 0)
    if (y <= height-1 && y >= 0)
      if (z <= slices-1 && z >= 0) {
        cudaPointsX[z*height*width+y*width+x] = newX;
        cudaPointsY[z*height*width+y*width+x] = newY;
        cudaPointsZ[z*height*width+y*width+x] = newZ;
      }
}
\end{lstlisting}

\begin{lstlisting}[label={code:occupancy}, caption={Funções usadas para calcular a ocupação máxima dos processadores da GPU}]
  int getBlockSize(int maxBlockSize) {
  int maxOccupancyBlockSize = 0;
  float maxOccupancy = 0.0;
  int device;
  cudaDeviceProp prop;
  cudaGetDevice(&device);
  cudaGetDeviceProperties(&prop, device);

  for (int blockSize = 32; blockSize <= maxBlockSize; blockSize += 32) {
    int numBlocks;        // Occupancy in terms of active blocks

    cudaOccupancyMaxActiveBlocksPerMultiprocessor(
        &numBlocks,
        tpsCuda,
        blockSize,
        0);

    int activeWarps = numBlocks * blockSize / prop.warpSize;
    int maxWarps = prop.maxThreadsPerMultiProcessor / prop.warpSize;
    float currentOccupancy = 1.0*activeWarps/maxWarps;
    if (currentOccupancy >= maxOccupancy ) {
      maxOccupancy = currentOccupancy;
      maxOccupancyBlockSize = blockSize;
    }
  }
  return maxOccupancyBlockSize;
}

dim3 calculateBestThreadsPerBlock(int blockSize, bool twoDim) {
  dim3 threadsPerBlock;
  std::vector<int> threadsPerDim(3, 1);
  int divisor = 8;
  int imageDimension;
  if (twoDim) {
    imageDimension = 2;
  } else {
    imageDimension = 3;
  }

  for (int i = 0; divisor > 1;) {
    if (blockSize%divisor == 0) {
      threadsPerDim[i%imageDimension] *= divisor;
      blockSize /= divisor;
      i++;
    } else {
      divisor /= 2;
    }
  }

  threadsPerBlock.x = threadsPerDim[0];
  threadsPerBlock.y = threadsPerDim[1];
  threadsPerBlock.z = threadsPerDim[2];

  return threadsPerBlock;
}
\end{lstlisting}