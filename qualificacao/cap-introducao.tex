%% ------------------------------------------------------------------------- %%
\chapter{Introdução}
\label{cap:introducao}
	O alinhamento espacial ou geométrico entre duas ou mais imagens é um problema clássico dentro de visão computacional. 
Ele leva pontos de uma imagem para pontos de outra imagem, criando uma função de correspondência entre imagens. Uma vez
alinhadas, uma série de aplicações distintas podem ser beneficiadas, dependendo do contexto do problema. 
A partir  dessa função podemos criar imagens maiores costurando imagens menores uma nas outras, processo esse chamado de 
\textit{stitching ou mosaicking}, utilizá-la para realizar a junção de imagens diferentes da mesma cena em uma única 
imagem, chamado de fusão (\textit{image fusion}), ou ainda extrair diferenças anatômicas a partir de imagens médicas.

	Chamamos o processo de alinhamento de imagens de registro de imagens. Os algoritmos de registro são categorizados 
baseando-se na natureza da transformação usada para alinhar as imagens. O primeiro grupo, o registro rígido, 
alinha as imagens aplicando transformações simples às imagens, como rotações e translações. Ele é pode
ser usado para corrigir movimentações simples, como a movimentação da cabeça de um paciente entre captura de imagens. Porém
essas transformações não são suficientes para corrigir movimentações fisiológicas mais complexas, como os batimentos do coração
ou o movimento respiratório. O registro desses casos é então realizado pelos algoritmos de registro não-rígido, 
que utilizam transformações não lineares para representar as deformações presentes nas imagens. O fluxo óptico ou a 
utilização de equações de dinâmica de fluidos para aproximar a transformação são exemplos de soluções para esses casos.

	O registro é altamente custoso computacionalmente por dois motivos. O primeiro é o número de etapas que devem ser
realizadas antes do algoritmo de registro em si. Dependendo da qualidade das imagens ou do algoritmo usado, todas as 
imagens devem passar por um pré-processamento. Depois, características são encontradas em cada imagem, e correspondências
são feitas entre características de imagens diferentes. O algoritmo de registro é então executado para
encontrar, dentro do espaço de parâmetros da transformação escolhida, o melhor conjunto que quando aplicado a transformação
nos dê o alinhamento espacial das imagens.

	O outro motivo que eleva o custo do registro é a grande quantidade de dados à disposição de pesquisadores atualmente,
dada a crescente facilidade em se manter e distribuir dados com os avanços na pesquisa com computação em nuvem, por exemplo.
Um destes exemplos são as imagens \textit{Gigapixel} \cite{cossairt2011gigapixel} compostas por milhões de pixels, criadas para 
representar, com a maior quantidade de detalhes possíveis, panoramas de cidades e até de microestruturas em 
estudos biológicos. Em 2010, o projeto criado por \cite{sevilla111}, que mantinha o recorde mundial com uma imagem de 111 
\textit{Gigapixels}, foi composto usando 9.750 imagens. Em apenas dois anos o projeto de \cite{london320} criou uma foto
esférica de Londres com 320 \textit{Gigapixels}, criada a partir de 48.640 imagens. Essa quantidade de imagens pode ser
facilmente encontrada em estudos dessa década em outras áreas de processamento de imagens.

	Logo o registro se beneficia de algum tipo de aceleração. O principal objetivo do trabalho é traduzir o registro de 
imagens para o modelo de execução \textit{Single Instruction, Multiple Data} \cite{patterson2013computer}, que é adotado 
nas Unidades de Processamento Gráfico (GPU). A escolha das GPUs como hardware para a execução do processo de registro vem da 
velocidade com que a sua capacidade de processamento aumenta, atrelado ao baixo custo de processamento dessas placas. Ao
escolher o algoritmo de registro que será utilizado temos que tomar o cuidado para que ele se adapte ao paradigma de 
programação em GPU a ponto de ganhar um aumento de velocidade considerável e que ainda sim seja capaz de registrar uma 
gama grande de imagens.

%% ------------------------------------------------------------------------- %%
\section{Trabalhos Relacionados}
\label{sec:objetivo}
	Desde que as Unidades de Processamento Gráfico começaram a ser usadas em meados dos anos 2002 para execução de 
código genérico, ou seja, sem a finalidade de renderizar uma cena, vários trabalhos traduziram algoritmos para a GPU.
Algoritmos esses que são aplicados em várias áreas, como cálculo numérico, simulações cientificas e processamento de 
imagens.

	Uma das primeira aplicações envolvendo o uso de GPUs no processamento de imagens foi feito por 
\cite{fung2005openvidia}, que descreve o arcabouço de programação \textit{OpenVIDIA}, criado a partir do 
\textit{OpenGL}, desenvolvido por \cite{opengl}. Antes do \textit{OpenVIDIA} o processamento de imagens em GPUs era
feito utilizando o \textit{OpenGL}, uma linguagem para processamento gráfico, voltada para a renderização de cenas. Os
algoritmos para processamento de imagem eram reescritos para utilizarem primitivas do \textit{OpenGL}. O 
\textit{OpenVIDIA} criou um encapsulamento de chamadas do \textit{OpenGL} voltado para imagens, retirando a necessidade
de traduzir os algoritmos e facilitando a utilização dos \textit{shaders} programáveis da GPU.

	Com a introdução de linguagens próprias para programação em GPU, vários algoritmos de processamento de imagens foram
traduzidos para a execução em GPUs. Na área de registro um dos primeiros artigos a tratar do assunto foi 
\cite{strzodka2004image}, que apresenta uma implementação de um algoritmo de registro não-rígido que utiliza uma técnica
de redução de energia da diferença das imagens. \cite{kohn2006gpu} expandiu o trabalho anterior, criando uma versão do 
algoritmo para imagens em três dimensões e realizando um registro rígido antes do não-rígido. Ainda programando 
diretamente em \textit{OpenGL}, \cite{vetter2007non} propõem um algoritmo de registro multimodal para imagens médicas 
executando em uma GPU 7800 da \textit{NVIDIA}. O artigo leva em conta a organização da imagem na memória da GPU e o 
tempo de transmissão dos dados da CPU para a GPU.

	O trabalho de \cite{grossauer2008gpu} relata a criação de um algoritmo para registro que utiliza fluxo óptico em 
conjunto com métodos de \textit{Multi Grid}, usados para resolver equações diferenciais parciais. Ele descreve
a aceleração do \textit{Multi Grid}, dada a sua afinidade com o \textit{Pipeline} gráfico. Em \cite{bui2009performance}
o passo de interpolação de um algoritmo de registro multi modal entre imagens médicas é traduzido utilizando uma 
linguagem para programação genérica em GPUs, o CUDA \cite{nvidia2007compute}. O estudo realiza uma comparação de 
eficiência entre o código em CPU e GPU de uma interpolação bilinear, apresentando como resultado uma aceleração de 60 
vezes da GPU em relação a CPU. O algoritmo de registro proposto por \cite{han2009gpu} utiliza Informação Mutua para
realizar um registro não-rígido entre imagens de ressonância magnética entre cérebros de indivíduos e um atlas com a 
finalidade de segmentar regiões de interesse no individuo. O estudo utiliza uma GPU \textit{NVIDIA} em conjunto com a 
linguagem CUDA para registrar as imagens, apresentando uma aceleração de 25 vezes em relação a CPU. Esse estudo, 
juntamente com o anterior, se preocupa com a organização dos dados na memória da GPU, algo que era de dificil controle 
antes do surgimento de linguagens voltadas para GPGPU.

	Em 2010, o artigo de \cite{modat2010fast} define um algoritmo de registro não-rígido chamado de \textit{Free-Form 
Deformation}, desenvolvido para registrar imagens de ressonância magnética pesadas em T1. Diferente de outros algoritmos, 
ele foi desenvolvido para executar em GPUs diretamente. Para tal, os autores utilizam uma interpolação B-Spline cúbica e 
uma modificação da informação mutua como métrica para o registro. Ele separa cara passo em vários \textit{kernels} diferentes, fazendo
com que cada \textit{kernel} utilize o máximo de memória possível e então espalha o processamento para um maior número
de processadores na GPU, melhorando o desempenho do algoritmo. 

	O estudo de \cite{membarth2011frameworks} realiza uma comparação entre vários arcabouços de programação para GPU,
entre eles o CUDA e o OpenCL utilizando como caso de comparação o registro de imagens 2D/3D, ou seja, o registro de
imagens em três dimensões de tomografia computadorizada para imagens de raio-X. O artigo então segue primeiro com uma
comparação conceitual entre os vários arcabouços e depois com uma comparação de desempenho. Foram feitas comparações 
com a execução sequencial, em paralelo e em GPU do algoritmo. 
	
%% ------------------------------------------------------------------------- %%
\section{Objetivos}
	Vários trabalhos na última década levaram o registro de imagens para o mundo GPGPU. Alguns deles implementam somente
algumas operações especificas do algoritmo em GPU, enquanto outros traduzem todo o processo do algoritmo. Porém o registro é composto de 
várias etapas auxiliares feitas antes do algoritmo principal, como etapas de pré-processamento, descoberta de 
características e correspondência das mesmas. Falaremos mais sobre isso no Capitulo \ref{cap:conceitos}. Ainda, a tradução 
proposta por eles não leva em conta a execução do mesmo algoritmo de registro repetidas vezes na GPU, ponto que pode ser
explorado para incrementar a velocidade do registro entre várias imagens.
	
	O objetivo principal do trabalho é identificar, entre os algoritmos mais utilizados para registrar imagens com 
deformações não-rígidas, o melhor candidato para ser portado para execução em GPUs. Os objetivos específicos deste
trabalho são:

\begin{enumerate}
	\item Identificar os algoritmos de registro não-rídidos mais comuns e implementá-los em GPU
	\item Otimizar a ocupação dos recursos da GPU, preferencialmente entre as etapas do registro, com a finalidade de 
				minimizar a transferência de dados entre a CPU e a GPU
	\item Lidar com o registro de mais de duas imagens por vez
	\item Avaliar a eficiência quando comparados à implementações em CPU
\end{enumerate}

%% ------------------------------------------------------------------------- %%
\section{Organização do Trabalho}
\label{sec:organizacao_trabalho}

No Capítulo~\ref{cap:conceitos}, apresentamos o que é Registro, Unidade de Processamento Gráfico, GPGPU e os algoritmos 
de registro estudados. No Capítulo \ref{cap:resultados} resultados preliminares da implementação de dois algoritmos
de registro em CPU são apresentados. Finalmente, no Capítulo~\ref{cap:conclusoes} analisamos os resultados obtidos. 